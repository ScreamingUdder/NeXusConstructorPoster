Neutron experiments can lead to the discovery of defects in plane parts and reveal interesting features about the composition of historical weapons. Such experiments involve directing an intense neutron beam at a material we wish to study and then examining the \textit{neutron scattering} that occurs as a result. This area of science has a wide variety of applications spanning physics, chemistry, biology, engineering and archaeology.

At the ISIS Pulsed Muon and Neutron Source we carry out 500 of these expriments every year. The European Spallation Source . These facilities comprise a neutron source which feeds tens of separate \textit{instruments} with neutrons with which to carry out different experiments in parallel. Neutron instruments are themselves made up of various components that manipulate the neutron beam, maintain the sample being studied in particular conditions, and detect the neutrons. These components can be used in a "mix-and-match" fashion depending on the nature of the experiment.

\iffalse
As different researchers will have different things they wish to investigate, the components that are used in an experiment and the way in which they're used will vary from experiment to experiment in a ``mix-and-match" fashion.

In order to analyse the results of neutron experiments reliably we need a record of the \textit{experiment configuration}. This means an accurate description of the detected neutrons and conditions of the sample, but also the precise geometry of the components in the neutron instrument which was used. This information is stored in NeXus files. The \textit{NeXus Constructor} is an application under development to allow scientists to define the instrument geometry and precisely what data should be recorded during an experiment.
\fi

