Neutron experiments can give many insights into materials such as indicating defects in plane parts and revealing interesting features about the composition of historical weapons. These experiments involve directing an intense neutron beam at a material we wish to study and then examining the \textit{neutron scattering} that occurs as a result of the neutrons colliding with the atoms in the material. This area of science has a wide variety of applications spanning physics, chemistry, biology, engineering and archaeology.

At the ISIS Pulsed Muon and Neutron Source we carry out around 500 of these expriments every year. A similar facility, the European Spallation Source, is presently under construction and will be able to generate $\sim$100x the amount of neutrons once it has been completed. These facilities comprise a neutron source which feeds tens of separate \textit{instruments} with neutrons which are then used to carry out different experiments in parallel. The neutron instruments are themselves made up of various components that manipulate the neutron beam, maintain the sample being studied in particular conditions, and detect the neutrons. 

