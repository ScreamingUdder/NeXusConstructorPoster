The NeXus file format arose out of a desire to record data from neutron experiments in a way that is ``facility-neutral." While this has its advantages, the format is not particularly easy to work with for the uninitiated. This can be especially cumbersome to scientists who would much rather get their data and get on with their day. The NeXus Constructor, a tool being developed by software developers based at ISIS Pulsed Muon and Neutron Source in the UK and the European Spallation Source (ESS) in Sweden, attempts to address this problem by providing an interface for scientists to easily examine and modify the contents of NeXus files.

\iffalse
Studying materials at the molecular level can lead to the discovery of defects in plane parts and reveal interesting features about the composition of historical weapons. One way of achieving this is to conduct a \textit{neutron experiment}. This entails directing an intense neutron beam at a material we wish to study and then examining the \textit{neutron scattering} that occurs as a result. Neutron experiments have a wide variety of applications spanning physics, chemistry, biology, engineering and archaeology.

ISIS Pulsed Muon and Neutron Source at the Rutherford Appleton Laboratory in the UK is currently collaborating on software with the European Spallation Source (ESS) which is being built in Sweden.

These facilities comprise a neutron source which feeds tens of separate \textit{instruments} with neutrons with which to carry out different experiments in parallel. Neutron instruments are themselves made up of various components that manipulate the neutron beam, maintain the sample being studied in particular conditions, and detect the neutrons. 

As different researchers will have different things they wish to investigate, the components that are used in an experiment and the way in which they're used will vary from experiment to experiment in a ``mix-and-match" fashion.

In order to analyse the results of neutron experiments reliably we need a record of the \textit{experiment configuration}. This means an accurate description of the detected neutrons and conditions of the sample, but also the precise geometry of the components in the neutron instrument which was used. This information is stored in NeXus files. The \textit{NeXus Constructor} is an application under development to allow scientists to define the instrument geometry and precisely what data should be recorded during an experiment.
\fi

