The NeXus file format arose out of a desire to describe the configurations of neutron, muon, and X-ray experiments in a way that is ``facility-neutral." A shared data format has a number of advantages such as reducing the need for conversion tools and fostering cooperation in software development across different facilities. However, the format is not particularly easy to work with for the uninitiated. This can be especially frustrating to scientists who would much rather get their data and get on with their day. The NeXus Constructor, a tool being developed by software developers based at ISIS Pulsed Muon and Neutron Source in the UK and the European Spallation Source (ESS) in Sweden, attempts to address this problem by providing an interface for users to easily examine and modify the contents of NeXus files.
