By focusing a neutron beam at a material and observing how the material responds to the beam, we are able to analyse the properties of materials on the atomic scale. This process is known as \textit{spallation} and has a variety of applications ranging from identifying the defects of plane parts to investigating the interior of historical artefacts without having to even open them. The instruments used to conduct these experiments are comprised of various different components that can be used to alter the beam neutrons or measure/detect certain phenomena. 

Such components include neutron detectors, which capture the pulse of energy after going through a target, and disk choppers, which interrupt the beam so that only neutrons of a certain speed may reach a target. Precise knowledge of where these components are place during an experiment is vital in order to analyse the output data reliably. 

The ISIS Pulsed Muon and Neutron source is one such facility and is a world-leader in terms of its scientific output. The European Spallation Source (ESS) is a similar facility that is presently under construction. As it will also be conducting neutron experiments, the decision was made to have it use software that is also used by ISIS. The NeXus Constructor arose out of a collaboration between software developers based at Rutherford Appleton Laboratory and the ESS in order to make a tool that allows scientists to examine and modify NeXus files that can be used to describe the configuration of neutron experiments.
