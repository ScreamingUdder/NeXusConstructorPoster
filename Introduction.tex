\iffalse
The NeXus file format arose out of a desire to describe the configurations of neutron, muon, and X-ray experiments in a way that is ``facility-neutral." However, the format is not particularly easy to work with for the uninitiated. This can be especially frustrating to scientists who would much rather get their data and get on with their day. The NeXus Constructor, a tool being developed by software developers based at ISIS Pulsed Muon and Neutron Source in the UK and the European Spallation Source (ESS) in Sweden, attempts to address this problem by providing an interface for users to easily examine and modify the contents of NeXus files.
\fi
Studying materials at the molecular level can help identify defects in plane parts and offer some insight into the composition of historical weapons. This is conducted by exposing a material to an intense neutron beam and studying the way in which the neutrons scatter. Neutron experiments have a wide variety of applications spanning physics, chemistry, biology, engineering and archaeology.

\iffalse ISIS Pulsed Muon and Neutron Source at the Rutherford Appleton Laboratory in the UK is currently collaborating on software with the European Spallation Source (ESS) which is being built in Sweden. \fi These facilities comprise a neutron source which feeds tens of separate \textit{instruments} with neutrons with which to carry out different experiments in parallel. Neutron instruments are themselves made up of various components to manipulate the neutron beam, maintain the sample being studied in particular conditions, and detect the neutrons. As different researchers will have different things they wish to investigate, the arrangement of components and different components can be used in a ``mix-and-match" fashion.

In order to analyse the results of neutron experiments reliably we need a record of the \textit{experiment configuration}. This means an accurate description of the detected neutrons and conditions of the sample, but also the precise geometry of the components in the neutron instrument which was used. This information is stored in NeXus files. The \textit{NeXus Constructor} is an application under development to allow scientists to define the instrument geometry and precisely what data should be recorded during an experiment.

