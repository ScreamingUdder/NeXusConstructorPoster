By exposing a material to an intense neutron beam we can gain a deeper understanding of its structure, even down to a molecular level. This has a wide variety of applications spanning physics, chemistry, biology, engineering and archaeology.

ISIS Pulsed Muon and Neutron Source at the Rutherford Appleton Laboratory in the UK is currently collaborating on software with the European Spallation Source (ESS) which is being built in Sweden. These facilities comprise a neutron source which feeds tens of separate \textit{instruments} with neutrons with which to carry out different experiments in parallel. Neutron instruments are themselves made up of various components to manipulate the neutron beam, maintain the sample being studied in particular conditions, and detect the neutrons.

The NeXus file format arose out of a desire to store all experiment data in a single file, in a format which is common to different neutron facilities, and which could be read by various software which scientists use to analyse the data. Required data includes not only records of the detected neutrons and conditions of the sample, but also the precise geometry of the components in the neutron instrument which was used. The \textit{NeXus Constructor} is an application under development to allow scientists to define the instrument geometry and precisely what data should be recorded during an experiment.
