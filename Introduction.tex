The NeXus file format arose out of a desire to record data from neutron experiments in a way that is ``facility-neutral." Despite its many benefits, the format is not particularly easy to work with for the uninitiated. This can be especially cumbersome to scientists who simply want to get their data and get on with their day. The NeXus Constructor, a tool being developed by software developers based at ISIS Pulsed Muon and Neutron Source in the UK and the European Spallation Source (ESS) in Sweden, attempts to address this problem by providing an interface for scientists to easily examine and modify the contents of NeXus files.
\iffalse
This means an accurate description of the detected neutrons and conditions of the sample, but also the precise geometry of the components in the neutron instrument which was used. This information is stored in NeXus files. The \textit{NeXus Constructor} is an application under development to allow scientists to define the instrument geometry and precisely what data should be recorded during an experiment.
\fi
