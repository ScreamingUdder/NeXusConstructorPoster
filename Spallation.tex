By bombarding certain types of metals with with high-energy protons can induce the release of neutrons. We can then investigate the molecular structure of a material by having these neutrons hit our material and observing how this causes the neutrons to scatter. This technique of harnessing neutrons is what is known as \textit{neutron spallation} and the experiments we are able to conduct through it have a number of significant applications ranging from analysing the composition of historical weapons to identifying defects in plane parts.

ISIS Pulsed Muon and 

Such experiments involve a mosaic of different components that are used to alter the beam of neutrons in some way and/or to measure/detect certain phenomena, and they will not all be used in the same exp in the same way. Such components include neutron detectors, which capture the pulse of energy after going through a target, and disk choppers, which periodically interrupt the beam so that only neutrons of a certain speed may reach the material that we wish to study. The sum of the components used, their properties, their position during the experiment, etc amounts to the \textit{experiment configuration}. A precise description of this configuration is vital in order to analyse our results reliably, hence the need for a file format that is appropriate for capturing all of this information.
