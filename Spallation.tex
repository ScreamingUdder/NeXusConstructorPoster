These neutrons can then be directed at a material so that we can observe how the material behaves in order to gain a deeper understanding of its molecular structure. This has a number of important applications ranging from analysing the composion of historial weapons to identifying defects in plane parts.

The instruments used to conduct spallation experiments are comprised of various different components that are used to alter the beam of neutrons and/or measure/detect certain phenomena. Such components include neutron detectors, which capture the pulse of energy after going through a target, and disk choppers, which interrupt the beam so that only neutrons of a certain speed may reach the material that we wish to study. Precise knowledge of the properties and whereabouts of these components is vital in order to analyse the results of our experiments reliably. 
