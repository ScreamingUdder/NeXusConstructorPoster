\textit{Neutron spallation} refers to the release of neutrons caused by bombarding certain types of metals with with high-energy particles. This allows us to conduct \textit{neutron experiments} by observing how the neutrons scatter upon colliding with a material of our choosing. In doing so, we may gain insight into the molecular structure of the material -- or sample -- responsible for causing the neutrons to scatter. Neutron experiments have a number of significant applications ranging from analysing the composition of historical weapons to identifying defects in plane parts.

At ISIS Pulsed Muon and Neutron source we carry out roughly 800 of these experiments every year. 

The facilities that carry out neutron experiments house a mosaic different components that are used to alter the beam of neutrons in some way and/or to measure/detect certain phenomena. These components include neutron detectors, which capture the pulse of energy after going through a sample, and disk choppers, which periodically interrupt the beam so that only neutrons of a certain speed can get through. As researchers have different things that they wish to study, these components will not all be used in every experiment and will not all be used in the same way. The sum of the components used during an experiment, their properties, their position, etc amounts to what we call the \textit{experiment configuration}. A precise description of this configuration is vital in order to analyse our results reliably, hence the need for a file format that is appropriate for capturing all of this information.
