By bombarding a metal target with an intense proton beam we can induce a release of neutrons. These neutrons can then be directed at a material of our choosing -- what is commonly called the experiment target or sample -- so that we may study the its molecular properties. The process of extracting these neutrons is what is known as \textit{neutron spallation} and spallation experiments have number of significant applications ranging from analysing the composition of historical weapons to identifying defects in plane parts.

Such experiments involve a mosaic of different components that are used to alter the beam of neutrons in some way and/or to measure/detect certain phenomena, and they will not all be used in the same exp in the same way. Such components include neutron detectors, which capture the pulse of energy after going through a target, and disk choppers, which periodically interrupt the beam so that only neutrons of a certain speed may reach the material that we wish to study. The sum of the components used, their properties, their position during the experiment, etc amounts to the \textit{experiment configuration}. A precise description of this configuration is vital in order to analyse our results reliably, hence the need for a file format that is appropriate for capturing all of this information.
