Neutron spallation involves the use of a high-power particle accelerator to fire a particle beam into a metal target, causing the target to release neutrons. We can then direct these neutrons at a material of our choosing in order to study its molecular structure. Neutron spallation experiments have a number of significant applications ranging from analysing the composition of historical weapons to identifying defects in plane parts.

Such experiments involve a mosaic of different components that are used to alter the beam of neutrons and/or measure/detect certain phenomena. Such components include neutron detectors, which capture the pulse of energy after going through a target, and disk choppers, which interrupt the beam so that only neutrons of a certain speed may reach the material that we wish to study. The sum of the components used, their properties, their position during the experiment, etc amounts to the \textit{experiment configuration}. A precise description of this configuration is vital in order to analyse our results reliably, hence the need for a file format that is appropriate for capturing all of this information.
