\mysubtitle{Features}
\begin{itemize}
\item \texttt{pint} - Unit conversion. This comes in handy when drawing 3D objects from mesh files or NeXus files that contain geometry information. As the STL and OFF mesh file formats do not come with unit information, through using pint we can at least allow the user to provide units when creating components so that objects to not become unexectedly large/small in the InstrumentView. Another advantage is that pint already knows that "metre", "meter", "metres", etc mean the same thing so we don't have to worry about writing code for handling these cases.
\item \texttt{h5py} - Reading from and writing to NeXus files.
\item \texttt{silx} - Displayng the contents of NeXus files in a tree view.
\end{itemize}
\mysubtitle{Development}
\begin{itemize}
\item \texttt{black} and \texttt{flake 8} - Code formatting and linting.
\item \texttt{pre-commit} 
\item \texttt{pytest} and \texttt{pytest-cov} - Testing with coverage output.
\item \texttt{pytest-qt} - Simulating button clicks and user input in order to check that the user clicking on object X triggers event Y. Especially useful when combined with \texttt{parametrize} to ensure that every \textit{possible} combination of actions that should lead to Y does indeed lead to Y.
\end{itemize}
